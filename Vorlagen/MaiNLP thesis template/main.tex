% Fristen: 
% https://www.cis.uni-muenchen.de/ba/termine/index.html
%
% Empfohlene Richtlinien für Bachelorarbeiten: 
% https://www.cis.uni-muenchen.de/ba/bachelorarbeit/richtlinien/index.html
%
% Informationen zu Abschlussarbeiten (Master):
% https://www.cis.uni-muenchen.de/master/masterarbeit/index.html

\documentclass[11pt,a4paper,twoside,openright]{scrbook}
\usepackage{clba}

% Per Kapitel Nummerierung von Graphiken und Tabellen
\usepackage{chngcntr}
\counterwithin{figure}{chapter}
\counterwithin{table}{chapter}


% Hier die eigenen Daten eintragen
\global\fach{Computerlinguistik}
\global\arbeit{Bachelorarbeit}
\global\titel{Titel der Arbeit}
\global\bearbeiter{Max Mustermann}
\global\betreuer{Dr. Max Mustermann}
\global\pruefer{Prof. Dr. Barbara Plank}
\global\universitaet{Ludwig- Maximilians- Universität München}
\global\fakultaet{Fakultät für Sprach- und Literaturwissenschaften}
\global\department{Department 2}

\global\abgabetermin{04. Juni 2012}
\global\bearbeitungszeit{27. März - 05. Juni 2023}
\global\ort{München}


\begin{document}

% Deckblatt
\deckblatt

\pagestyle{scrheadings}
\pagenumbering{gobble}

% Erklärung fürs Prüfungsamt
\erklaerung

% Zusammenfassung
\addchap{Abstract}
\thispagestyle{scrplain}
\noindent
Dieses 
dient als Muster für eine Ausarbeitung einer
Bachelorarbeit am CIS und wird in deutscher oder englischer Sprache
erstellt (hier max. 250 Wörter)

% Inhaltsverzeichnis
\pagenumbering{Roman}

\tableofcontents

% Text mit arabischer Nummerierung
\pagenumbering{arabic}

\chapter{Kapitel Eins}

\section{Ein Abschnitt}
Mein Name ist Hase und ich weiß von nichts. Das ist ein Testtext. Mein Name ist
Igel und ich weiß auch von nichts. 

\vspace{1em}
\noindent Zitationen mit natbib:\footnote{\url{https://ctan.org/pkg/natbib?lang=en}} 

\begin{itemize}
    \item \textit{\textbackslash cite\{vaswani2017attention\}} $\overrightarrow{}$ \cite{vaswani2017attention}
    \item \textit{\textbackslash citep\{vaswani2017attention\}} $\overrightarrow{}$ \citep{vaswani2017attention}
    \item \textit{\textbackslash citet\{vaswani2017attention\}} $\overrightarrow{}$ \citet{vaswani2017attention}
\end{itemize}


\subsection{Ein Unterabschnitt}
Blabla. Hier ein Unterabschnitt.

\subsubsection{Ein Unterunterabschnitt}
\label{sec:a}
Blabla. Hier ein Unterunterabschnitt.

\subsubsection{Noch ein Unterunterabschnitt}
\label{sec:b}

Wer \ref{sec:a} sagt, muss auch \ref{sec:b} sagen.

\subsection{Noch ein Unterabschnitt}

Das ist ein gewöhnlicher Absatz.

\paragraph{Ein Absatz mit Titel}
Paragraphen gibts auch.

\subparagraph{Ein Unterabsatz mit Titel}
Und dann auch noch Unterparagraphen.

\subsection*{Ein nicht nummerierter Unterabschnitt}
Dieser Unterabschnitt erscheint nicht im Inhaltsverzeichnis.
\newpage

\section{Beispiele}
Blabla.
\newpage

\section{Mehr Beispiele}
Blabla.
\newpage

\input{80k_characters}
 %\input{120k_characters}

%Beispielliteratur
\bibliography{references}
\bibliographystyle{acl_natbib}


% \begin{thebibliography}{9}
% \bibitem{Erdos01} P. Erd\H os, \emph{A selection of problems and
% results in combinatorics}, Recent trends in combinatorics (Matrahaza,
% 1995), Cambridge Univ. Press, Cambridge, 2001, pp. 1--6.

% \bibitem{ConcreteMath}
% R.L. Graham, D.E. Knuth, and O. Patashnik, \emph{Concrete
% mathematics}, Addison-Wesley, Reading, MA, 1989.

% \bibitem{Knuth92} D.E. Knuth, \emph{Two notes on notation}, Amer.
% Math. Monthly \textbf{99} (1992), 403--422.

% \bibitem{Simpson} H. Simpson, \emph{Proof of the Riemann
% Hypothesis},  preprint (2003), available at 
% \url{http://www.math.drofnats.edu/riemann.ps}.
% \end{thebibliography}
\newpage

% Abbildungsverzeichnis (kann auch nach dem Inhaltsverzeichnis kommen)
\listoffigures
\newpage

% Tabellenverzeichnis (kann auch nach dem Inhaltsverzeichnis kommen)
\listoftables
\newpage

\addchap{Inhalt des beigelegten Software/Datenpackets}

\end{document}