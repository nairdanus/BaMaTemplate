\documentclass[11pt,a4paper,twoside,openright]{scrbook}

\usepackage[master, german]{BaMa}


\global\titel{Titel}


\begin{document}

\pagenumbering{gobble}
\deckblatt
\pagestyle{scrheadings}

% Inhaltsverzeichnis
\pagenumbering{Roman}
\tableofcontents

% Text mit arabischer Nummerierung
\clearpage
\pagenumbering{arabic}


\chapter{Allgemeine Hinweise}

\section{Einleitung}
Dieses Dokument dient als Vorlage für Bachelor- bzw. Masterarbeiten\footnote{Die entsprechend kommentierten Blöcke im Bereich der Titelseite sowie die abschließenden Seiten (Lebenslauf und Erklärung) sind bei einer Hausarbeit in der \texttt{tex-Datei} zu entfernen.} des Studiengangs 'Computerlinguistik' (LMU München). Es basiert auf einem LaTeX\footnote{LaTeX kann hier heruntergeladen werden: \url{https://www.latex-project.org/}; nach der Installation kann die \texttt{tex-Datei} mit einem LaTeX-Editor, z. B. TeXShop (Mac, bereits mitinstalliert) oder Texmaker (Windows, muss zusätzlich installiert werden) geöffnet werden.}-Template, das auf der \href{https://www.cis.lmu.de/}{CIS-Homepage} erhältlich ist.

Die Bibliographie wird mit Hilfe von BibTeX generiert (s. \verb+bib-Datei+).


\section{Hinweise zur Form}

\subsection{Zitierung} 
\label{sec:cit}

Die Zitierweise folgt dem Autor-Jahr-System. Der Stellenverweis erfolgt im Text, Fußnoten sollten nur für Erläuterungen und Kommentare verwendet werden. Folgende Hinweise sind der \emph{Language}-Bibliographie-Vorlage von \url{ron.artstein.org} entnommen:\newline

\noindent
The \emph{Language} style sheet makes a distinction between two kinds of in-text citations: citing a work and citing an author.
\begin{itemize}
\item Citing a work:
  \begin{itemize}
    \setlength{\itemsep}{0pt}
    \setlength{\parsep}{0pt}
  \item Two authors are joined by an ampersand (\&).
  \item More than two authors are abbreviated with \emph{et al.}
  \item No parentheses are placed around the year (though parentheses
    may contain the whole citation). 
  \end{itemize}
\item Citing an author:
  \begin{itemize}
    \setlength{\itemsep}{0pt}
    \setlength{\parsep}{0pt}
  \item Two authors are joined by \emph{and}.
  \item More than two authors are abbreviated with \emph{and colleagues}.
  \item The year is surrounded by parentheses (with page numbers, if
    present).
  \end{itemize} 
\end{itemize}
To provide for both kinds of citations, \verb+language.bst+ capitalizes on the fact that \verb+natbib+ citation commands come in
two flavors. In a typical style compatible with \verb+natbib+, ordinary commands such as \verb+\citet+ and \verb+\citep+ produce short
citations abbreviated with \emph{et al.}, whereas starred commands such as \verb+\citet*+ and \verb+\citep*+ produce a citation with a
full author list. Since \emph{Language} does not require citations with full authors, the style \verb+language.bst+ repurposes the starred commands to be used for citing the author. The following table shows how the \verb+natbib+ citation commands work with \verb+language.bst+.

Authors of \emph{Language} articles would typically use \verb+\citet*+, \verb+\citep+, \verb+\citealt+ and \verb+\citeauthor*+, though they
could use any of the above commands. There is no command for giving a full list of authors.

\subsection{Hinweis Literaturverzeichnis}
Das Literaturverzeichnis dieser Vorlage beinhaltet als Beispielbibliographie die Literaturangaben des \emph{Language} Stylesheets. 







\chapter{Tipps zu \LaTeX}

\section{Character}
\begin{table}[h]
  \centering
  \begin{tabular}{lc}
  \hline
  \textbf{Command} & \textbf{Output}\\
  \hline
  \verb|{\"a}| & {\"a} \\
  \verb|{\^e}| & {\^e} \\
  \verb|{\`i}| & {\`i} \\ 
  \verb|{\.I}| & {\.I} \\ 
  \verb|{\o}| & {\o} \\
  \verb|{\'u}| & {\'u}  \\ 
  \verb|{\aa}| & {\aa}  \\\hline
  \end{tabular}
  \begin{tabular}{lc}
  \hline
  \textbf{Command} & \textbf{Output}\\
  \hline
  \verb|{\c c}| & {\c c} \\ 
  \verb|{\u g}| & {\u g} \\ 
  \verb|{\l}| & {\l} \\ 
  \verb|{\~n}| & {\~n} \\ 
  \verb|{\H o}| & {\H o} \\ 
  \verb|{\v r}| & {\v r} \\ 
  \verb|{\ss}| & {\ss} \\
  \hline
  \end{tabular}
  \caption{Example commands for accented characters, to be used in, \emph{e.g.}, Bib\TeX{} entries.}
  \label{tab:accents}
  \end{table}


% \section{Document Body}

% \subsection{Footnotes}

% Footnotes are inserted with the \verb|\footnote| command.\footnote{This is a footnote.}

% \subsection{Tables and figures}

% See Table~\ref{tab:accents} for an example of a table and its caption.
% \textbf{Do not override the default caption sizes.}

% \subsection{Hyperlinks}

% \verb|\hyperref{}|

% \subsection{Citations}

% Table~\ref{citation-guide} shows the syntax supported by the style files.
% We encourage you to use the natbib styles.
% You can use the command \verb|\citet| (cite in text) to get ``author (year)'' citations, like this citation to a paper by \citet{Gusfield:97}.
% You can use the command \verb|\citep| (cite in parentheses) to get ``(author, year)'' citations \citep{Gusfield:97}.
% You can use the command \verb|\citealp| (alternative cite without parentheses) to get ``author, year'' citations, which is useful for using citations within parentheses (e.g. \citealp{Gusfield:97}).

% \subsection{References}

% \nocite{Ando2005,augenstein-etal-2016-stance,andrew2007scalable,rasooli-tetrault-2015,goodman-etal-2016-noise,harper-2014-learning}

% The \LaTeX{} and Bib\TeX{} style files provided roughly follow the American Psychological Association format.
% If your own bib file is named \texttt{custom.bib}, then placing the following before any appendices in your \LaTeX{} file will generate the references section for you:
% \begin{quote}
% \begin{verbatim}
% \bibliographystyle{acl_natbib}
% \bibliography{custom}
% \end{verbatim}
% \end{quote}
% You can obtain the complete ACL Anthology as a Bib\TeX{} file from \url{https://aclweb.org/anthology/anthology.bib.gz}.
% To include both the Anthology and your own .bib file, use the following instead of the above.
% \begin{quote}
% \begin{verbatim}
% \bibliographystyle{acl_natbib}
% \bibliography{anthology,custom}
% \end{verbatim}
% \end{quote}
% Please see Section~\ref{sec:bibtex} for information on preparing Bib\TeX{} files.

% \subsection{Appendices}

% Use \verb|\appendix| before any appendix section to switch the section numbering over to letters. See Appendix~\ref{sec:appendix} for an example.

% \section{Bib\TeX{} Files}
% \label{sec:bibtex}

% Unicode cannot be used in Bib\TeX{} entries, and some ways of typing special characters can disrupt Bib\TeX's alphabetization. The recommended way of typing special characters is shown in Table~\ref{tab:accents}.

% Please ensure that Bib\TeX{} records contain DOIs or URLs when possible, and for all the ACL materials that you reference.
% Use the \verb|doi| field for DOIs and the \verb|url| field for URLs.
% If a Bib\TeX{} entry has a URL or DOI field, the paper title in the references section will appear as a hyperlink to the paper, using the hyperref \LaTeX{} package.




% Literatur
\printbibliography
\newpage


% Erklärung fürs Prüfungsamt
\pagenumbering{gobble}
% \erklaerung

\end{document}